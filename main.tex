%-----------------------------------------------------------------------------
%
%               Template for sigplanconf LaTeX Class
%
% Name:         sigplanconf-template.tex
%
% Purpose:      A template for sigplanconf.cls, which is a LaTeX 2e class
%               file for SIGPLAN conference proceedings.
%
% Guide:        Refer to "Author's Guide to the ACM SIGPLAN Class,"
%               sigplanconf-guide.pdf
%
% Author:       Paul C. Anagnostopoulos
%               Windfall Software
%               978 371-2316
%               paul@windfall.com
%
% Created:      15 February 2005
%
%-----------------------------------------------------------------------------


\documentclass[preprint]{sigplanconf}

% The following \documentclass options may be useful:

% preprint      Remove this option only once the paper is in final form.
% 10pt          To set in 10-point type instead of 9-point.
% 11pt          To set in 11-point type instead of 9-point.
% authoryear    To obtain author/year citation style instead of numeric.

\usepackage{amsmath}

\usepackage{minted} 			% Code listing with reasonable good highlighting
\usepackage[no-math]{fontspec}
\setmonofont[Scale=MatchLowercase]{Apple Symbols} % Unicode mono font


\begin{document}

\special{papersize=8.5in,11in}
\setlength{\pdfpageheight}{\paperheight}
\setlength{\pdfpagewidth}{\paperwidth}

\conferenceinfo{CONF 'yy}{Month d--d, 20yy, City, ST, Country} 
\copyrightyear{20yy} 
\copyrightdata{978-1-nnnn-nnnn-n/yy/mm} 
\doi{nnnnnnn.nnnnnnn}

% Uncomment one of the following two, if you are not going for the 
% traditional copyright transfer agreement.

%\exclusivelicense                % ACM gets exclusive license to publish, 
                                  % you retain copyright

%\permissiontopublish             % ACM gets nonexclusive license to publish
                                  % (paid open-access papers, 
                                  % short abstracts)

\titlebanner{banner above paper title}        % These are ignored unless
\preprintfooter{short description of paper}   % 'preprint' option specified.

\title{Generic diff3 for algebraic datatypes}
\subtitle{Subtitle Text, if any}

\authorinfo{Marco Vassena}
           {Chalmers University}
           {vassena@chalmers.se}
\authorinfo{Name2\and Name3}
           {Affiliation2/3}
           {Email2/3}

\maketitle

\begin{abstract}

\end{abstract}

\category{CR-number}{subcategory}{third-level}

% general terms are not compulsory anymore, 
% you may leave them out
\terms
term1, term2

\keywords
keyword1, keyword2

\section{Introduction}

\section{Universe}
In order to implement data type generic algorithms, we need a generic representation of algebraic data types.
The universe used in this paper consists of typed heterogeneous rose trees,
a combination of mutually recursive heterogeneous lists and trees
whose nodes corresponds to data type constructors and their subtrees
to their fields.
\begin{minted}{agda}
  data HTree : Set -> Set where
    Node : F as a -> HList as -> HTree a

  data HList : List Set -> Set where
    [] : HList []
   _∷_ : HTree a -> HList as -> HList (a ∷ as)
\end{minted}
	The type \texttt{F as a} denotes the constructor of an algebraic data
	type of type \texttt{a} that takes arguments of types determined by the list 
	\texttt{as}. This representation is type-safe because the constructor
	\texttt{Node} requires the same types \texttt{as} in the fields types and in 
	the list of subtrees, hence denoting a 	well-typed application of a 
	constructor to its arguments.
	Its definition is unimportant for the moment, therefore we will leave it
	abstract using a postulate. Furthermore a basic comparison
	function is assumed.
\begin{minted}{agda}
  postulate F : List Set -> Set -> Set 
  postulate _=?=_ : (α : F as a) (β : F bs b) -> Dec (α ≡ β)
\end{minted}	
	In section \ref{Haskell} we show how it is possible to implement these 
	features in a strongly typed functional language like Haskell.
	
	In the following we will use two auxiliary functions to append and split 
	heterogeneous lists:
\begin{minted}{agda}
_+++_ : HList as -> HList bs -> HList (as ++ bs)
split : ∀ {{as bs}} -> HList (as ++ bs) -> HList as × HList bs
\end{minted}
	
\section{Generic diff}
	The \texttt{diff} algorithm used as a subroutine in \texttt{diff3} is a
	generalization of that of Lempsink et al. \cite{Lemp09}, therefore
	we only briefly cover the aspects relevant for the description of
	the \texttt{diff3} algorithm.
	The algorithm takes two heterogeneous lists as arguments and
	outputs an edit scripts that details the differences between them,
	specifically with a list of edit operations that progressively 	
	transform one into the other.

	\subsection{Edit Operations}
	Single edits are defined over values, which denote either the presence or 
	absence of nodes. 
\begin{minted}{agda}
data Val : List Set -> List Set -> Set where
  ⊥ : Val [] []
  ⟨_⟩ : F as a -> Val as [ a ] 
\end{minted}
	Values are indexed by two lists that are used to stack edits in a type 
	safe manner and	that contain respectively the types of the fields of a node 
	and its resulting type.
	Since empty values do not store any node their lists are both empty.

	The set of edit operations considered includes a no-operation, node 
	deletion, insertion and update.
\begin{minted}{agda}
data _~>_ : Val as bs -> Val cs ds -> Set where
  Nop : ⊥ ~> ⊥
  Del : (α : F as a) -> ⟨ α ⟩ ~> ⊥
  Ins : (α : F as a) -> ⊥ ~> ⟨ α ⟩
  Upd : (α : F as a) (β : F bs a) -> ⟨ α ⟩ ~> ⟨ β ⟩
\end{minted}
	Edits are indexed by two values, called respectively 
	\emph{source} and \emph{target} values. The \texttt{Nop} edit is 
	a no-operation that is used exclusively to \emph{align} edit scripts,
	therefore it maps the empty value into itself.
	The edit \texttt{Del α} deletes the node \texttt{α} from the source object, 
	thus it maps it into \texttt{⊥}. Conversely the edit \texttt{Ins α} 
	inserts the node \texttt{α} in the target object and therefore has source 
	\texttt{⊥}.
	Lastly \texttt{Upd α β} denotes the update of the source node \texttt{α}
	to the target node \texttt{β}, which concretely represents a 
	constructor change. 
	Note that there is not a distinct copy edit, which is instead encoded as a 
	special	update, in which source and target nodes are the same.
	Edits whose source and target values are identical are 
	called \emph{identity} edits. Copies and no-operations fall into this
	category.
	
	\subsection{Edit Script}
	An edit script collects a finite number of edit operations, 
	while preserving type-safety. 
\begin{minted}{agda}
data ES : List Set -> List Set -> Set where
  [] : ES [] []
  _::_ : {v : Val as bs} {w : Val cs ds} -> v ~> w -> 
          ES (as ++ xs) (cs ++ ys) -> ES (bs ++ xs) (ds ++ ys)
\end{minted}	
	An edit script of type \texttt{ES xs ys} transforms
	a \emph{source} object of type \texttt{HList xs} into
	a \emph{target} object of type \texttt{HList ys}. 
	The following function retrieves the source object from an edit script:			
\begin{minted}{agda}
⟪_⟫ : ES as bs -> HList as
⟪ [] ⟫ = []
⟪ Nop ∷ e ⟫ = ⟪ e ⟫
⟪ Del α ∷ e ⟫ with hsplit ⟪ e ⟫
... | hs₁ , hs₂ = Node α hs₁ ∷ hs₂
⟪ Ins α ∷ e ⟫ = ⟪ e ⟫
⟪ Upd α β ∷ e ⟫ with hsplit ⟪ e ⟫
... | ds₁ , ds₂ = Node α hs₁ ∷ hs₂
\end{minted}
	No-operations and insert edits do not affect the source object,
	because their source vale is empty, therefore in these cases the source 
	function is simply called recursively.
	On the other hand delete and update edits alter a node of the
	source object by either removing it or changing it.
	In these two cases such node is output as part of the source object.
	The source function is called recursively on the rest of the edit 
	script producing a list, whose prefix correspond to the subtrees of the
	top node. Using the function \texttt{hsplit} the list is decomposed 
	accordingly in two parts, which are then assembled appropriately.
	
	Dually we define a function that retrieves the target object.
	Since it is analogous to the previous we give only its signature:
	
\begin{minted}{agda}
⟦_⟧ : ES as bs -> DList bs
\end{minted}

	\subsection{Diff algorithm}

\section{Generic diff3}

\section{Properties}

\section{Discussion and Related Work}

\appendix
\section{Appendix Title}

This is the text of the appendix, if you need one.

\acks

Acknowledgments, if needed.

% We recommend abbrvnat bibliography style.

\bibliographystyle{abbrvnat}

% The bibliography should be embedded for final submission.

\begin{thebibliography}{}
\softraggedright

\bibitem[Smith et~al.(2009)Smith, Jones]{smith02}
P. Q. Smith, and X. Y. Jones. ...reference text...

\end{thebibliography}


\end{document}

%                       Revision History
%                       -------- -------
%  Date         Person  Ver.    Change
%  ----         ------  ----    ------

%  2013.06.29   TU      0.1--4  comments on permission/copyright notices

